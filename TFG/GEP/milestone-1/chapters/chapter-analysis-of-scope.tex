\chapter{Analysis of scope}

\section{Scope}

The project will mainly consist of three separate phases of development. As this work will follow agile principles, each phase will consist of short sprints of condensed work, and the general flow of the project may not be entirely linear. For example, After interviewing some players, the Developer might detect a design problem in the game, and they will go back to phase 2 for solving the problem.

\subsection{Prototype development}

Based on current research, Various prototypes for the final game will be developed. During this phase player feedback and rapid iteration will be key for sorting out non-working prototypes and for finding the best candidate for the next phase. 

\subsection{Final game development}

This phase will transform the best prototype out of all the ones developed in the previous section into a fully-fledged, ready to be distributed mobile game. During this phase, constant communication with the thesis supervisor and some NGO's interested in the project will be very important for bringing in ideas and weight out features to add to the final work. 

\subsection{Player interviewing}

Once the final game is tested and ready to be deployed, this phase will begin. Through collaboration with other NGO's, a series of player interviews will be done. Conclusions about the effectiveness of the game for transmitting the ideas laid out by the project's objectives.

\section{Risks}

The risks are of different nature depending on which phase of the project we focus our attention on.

On the first phase, there exists the risk of not having developed prototypes of adequate quality or that don't reflect the research they were inspired from. To help minimize the possibility of this happening, several interviews with the thesis supervisor will be scheduled during this phase reporting the initial prototype proposals to them.

During the second phase, there exists a risk of developing a skewed final version of the first prototype that does not meet the required standards for the next phase or for distribution. A close observation of the development by the thesis supervisor and the use of an adequate project management methodology will help mitigate this risk.

During the third phase, it can happen that a shortage of time keeps the interviews from developing in a satisfactory manner. An adequate allocation of time will be sufficient to eliminate the risk. Another threat to consider is the fact that there may be errors in the source code that stretch out development time. To control this, thorough tests will be conducted during the development phase to help ensure the final game is shipped without serious errors in the source code. 

\section{Methodology and rigor}

The project development methodology that bests suits all the project's phases is Scrum. Scrum, a methodology based on Agile ideas of development, will help obtain a good communication between the thesis supervisor and the developer.

Scrum is an iterative framework for managing product development. A key principle of Scrum is the recognition that customers will change their minds about what they want or need and that there will be unpredictable challenges for which a planned approach is not suited. As such, Scrum accepts that the problem cannot be fully understood or defined up front, and instead focusing on responding to emerging requirements.

Scrum can be used for coordinating research work, developing software or even plan weddings
% Source
. It is because its flexibility and its focus on iteration and rapid prototyping that this methodology suits the project the best.


\section{Project management tools}

Several management tools will be used for this project. Below are described the principal ones.

\begin{description}
\item[Trello]{Trello is an organization tool. It mimics the Scrum board philosophy, and offers the user a virtual version of a board and post-it notes so often used when working with this framework. }
\item[Unity Collaborate]{ A git-based version control service built into Unity. }
\item[Google Calendar]{ Google's virtual calendar, useful for integrating with Trello and other services and to handle deadlines. }
\item[Slack]{ A Slack group will be used for beta testers to post their bug reports and other feedback regarding the final game. }
\end{description}

\section{Monitoring}

There will be a regular entry posting to the development log in itch.io\footnote{wextia.itch.io/rewild/devlog}, as a way for writing down findings or problems occurred during development and informing the thesis supervisor and beta testers of the project's development progress. 

With this and the other collaboration tools, an adequate monitoring of the project is ensued.