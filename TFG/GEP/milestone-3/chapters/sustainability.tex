\chapter{Sustainability}

Previous to the analysis on sustainability, it will be introduced an answer
to the survey filled by the thesis author.

\say{
  I feel fairly confident about assessing wether a technological project is
  sustainable, in all of its aspects.

  I am familiar with the
  techniques for developing a project in a sustainable manner and for identifying
  problems with an existing one.

  I strongly feel that every project
  one participates on should strive for the benefit of society in one way or
  another as opposed to personal gain or enrichment.   

  As for the economical part, i feel that a
  project should generate revenue as a consequence for its positive impact
  on society and of its efficiency when dealing with the environment.
  Every endeavour should be initiated with only that in mind.

  Collaborative work tools are very familiar to me, as i have used them 
  extensively, and still do. I feel that collaboration tools have 
  a great impact on how efficiently a project is developed, and i
  particulary admire their capabilities of bringing together teams of
  various ethnicities and culture backgrounds to the office through 
  video-conferences via Internet.

  As far as professional ethics go, i am familiarized with the Spanish IT
  professionals' ethics code
  \footnote{http://cpiiand.es/wordpress/download/CPIIA-Codigo\_Deontologico-fCPIIA.pdf}
  . I've read it and i apply it as much as i can on my dailiy work.

  I am very concerned about sustainability in and out of my work, but as
  a kind of collateral aspect to my personal ethics. I personally don't 
  believe that, for example, respect for other people's ethnicity should
  occur because of the notion of "social sustainability", but because it
  is the \emph{correct} way of treating people. I don't think we should
  respect nature because we \emph{need} it, but because it is the 
  \emph{correct} way of treating it. Looking for sustainability in a project
  just means that it is wanted for the project to last a long time. I am
  more concerned about making ethical projects. I don't really care if they
  don't last a lot as long as they serve their purpose to its fullest.
}


\section{Economical area}

The information for the economical area and its inpact on the project's sustainability rating is extracted from chapter \ref{economical}.

\section{Social area}

Gaming and education have proven to be a difficult couple to pair. It has been demonstrated that simulations and videogames can vastly improve the learning experience of a person
% cite
, but its full potential is far from being achieved. Currently, much of the education methods currently in use are branches of the British academic education method from the 18th century
% source
.

A project like this one may open the doors for a richer understanding of gaming as a key stone for learning in Spain, and as it will be distributed on worldwide markets, it could spread worldwide.

On a more realistic scale, the project aims to improve the people's ability to understand nature and sustainability, which yield a more cultured and informed group of people just for the sake of having played a game. The fact that people will get enjoyment out of this project as much as they will get an education is in itself a very positive impact the project will have if we look at it from a social prespective. 

As for the personal gain extracted from this project, it is expected that
all the people invested in this project gain further understanding about
nature and our relationship with it, and that they gain confidence in 
videogames as a useful tool for learning other than for entertainment only.

\section{Environmental area}

This project is centered around environmental sustainability. Its aims are to teach environmental sustainability and respect for nature using modern and effective techniques. Because of this, the final goal of the project is to make all players take into consideration their actions with nature in mind. This, in of itself, is very positive for the environmental sustainability of the project. 

\section{Conclusions}

A table summarizing this project's scores on the sustainability matrix will be included below. It is important to note that it is incomplete, as as the project advances more of the matrix will be possible to fill.

\begin{center}
  \begin{tabular}{ | c | c | c | c | }
    \hline
         &
        \textbf{Economical area} &
        \textbf{Social area} &
         \textbf{Environmental area} \\ 
        \hline
        Planning & 7 & 9 & 10 \\  
        \hline
  \end{tabular}
\end{center}

As this project is not thought of to generate revenue, the economic area is rated lower than the other ones. 
