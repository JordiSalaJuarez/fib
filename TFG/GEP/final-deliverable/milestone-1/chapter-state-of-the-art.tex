\chapter{State of the art}

\section{Research on the subject}

There are numerous papers 
\cite{gamingnaturalresource}
\cite{roleplayingasatool}
\cite{usingsimulation}
that work on the subject of serious games about education for sustainability. The majority of the work on this field relies on more traditional gaming experiences, such as board games or role playing experiences in the classroom or via interviews with real life stakeholders.

\section{Video games on the market}

Several games in the mobile market loosely follow the tone of this project, based on a keyword search on the Android Play Store, we can extract a collection of games that are valuable to get inspiration from based on the project's objectives.

Although a lot more games have been downloaded and played by the Researcher, only a handful have been found to be related with the project's objectives. Below it can be found a list of those games with an analysis of it in the context of the project.

\begin{description}

\item[Pixel Farm]{Farming simulator game played in \gls{portraitmode} with a \gls{pixelart}
aesthetic. Some mechanics about planting and harvesting seem interesting.}
\item[My Oasis]{Zen game about growing a garden in your mobile device. It is a \gls{clicker} game with \gls{lowpoly} 3D graphics. The player experiment a bonding experience with their tiny simulation, and can be a viable way to communicate with them.}

\item[Dont Starve]{It is a \gls{survivalgame} about resisting the harshness of nature and various monster attacks. Some mechanics like exploration and the dynamics of creature \glspl{npc} might be useful for the project.}

\item[Animal Crossing Pocket Camp]{A simplified version of the original Animal Crossing series for mobile devices. An industry giant with a very high ammount of polish. The general tone of the game generally is harmonious with nature, and its mechanics can be used to convey a sense of connection with nature}

\item[Desertopia]{An \gls{idle} \gls{clicker} game that consists of watering a desert to help nature in it sprout again. The general idea behind it aligns nicely with the objectives of the project and some ideas and mechanics could be used for it.}

\item[Pocket Plants]{A garden simulator game with a cartoon-y art style. Some of its mechanics and tone might be useful for gaming experiences oriented towards children.}

\item[Greening 2]{An \gls{idle} game about terraforming tiny planets by absorbing water and several other elements from asteroids and using them to fertilize the planet's soil. Very in depth with its mechanics and achievments, might be worthy to further explore for implementing a \gls{lategame} \gls{progressionsystem}.}

\end{description}

\section{Game Engines}

Different Game Engines were considered for developing this project. Amongst  all possible solutions, the Unity game engine was chosen for this project.

The reasons for this choice are the flexibility and extensibility of the engine, the affordable price for personal use (Free until the project receives 100 thousand dollars in funding or revenue) and its powerful capabilities for multi platform development.  

\section{Conclusions}

Despite extensive research in the subject, it is evident that not much of it has been applied to mobile gaming experiences. The project will be developed with one of the most popular and well-tested Game Engines in the market, and having in mind various mobile video games that are somewhat related to the project's objectives. All of this is done in order to ensure that the idea and the implementation of it on a mobile device is executed having in mind all the possibilities, both from the point of view of academic research and of mobile player-game interaction for obtaining interesting and applicable results.
