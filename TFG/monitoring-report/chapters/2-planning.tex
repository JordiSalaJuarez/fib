\chapter{Planning}

\section{Changes regarding original planning}

Regarding original planning, we found that 4 weeks for developing the final game was not enough. During the GEP course, the developer worked outside its allotted time to make the prototypes. This yielded a bigger time frame to work with for the developer once GEP finished. This means that instead of
4 weeks for developing the final game, we ended up with 8 weeks instead, which we consider sufficient to work with.


\begin{figure}[h]
\caption{Resulting time table adjusted to schedule deviations}
\centering
\includegraphics[width=\textwidth]{figures/planification}
\label{fig:planning}
\end{figure}

\subsection{Affectations on the budget}

Deviations, of course, affect the final budget. Below there can be found a table detailing those changes.

\begin{center}
    \begin{tabular}{ | l | r | r | r | }
        \hline
        \textbf{Role} & \textbf{Hours} & \textbf{Hourly wage} & \textbf{Total} \\ 
        \hline
        \hline
        Project Manager & 70 & 50,00€ & 3500,00€ \\  
        Researcher & 60 & 35,00€ & 2100,00€ \\
        Software developer & 220 & 20,00€ & 4400,00€ \\
        Interviewer & 40 & 15,00€ & 600,00€ \\
        Quality Asssurance & 40 & 20,00€ & 800,00€ \\
        \hline
        \textbf{Total} &  &  & 11400,00€ \\      
        \hline
    \end{tabular}
\end{center}

\begin{center}
    \begin{tabular}{ | l | r | }
        \hline
        \textbf{Concept} & \textbf{Price} \\
        \hline
        \hline
        Human Resources & 11400,00€ \\  
        Hardware & 115,59€ \\  
        Software & 693,00€ \\  
        Licensing & 29,38€ \\
        Indirect spending & 226,95€ \\
        \hline
        \textbf{Total} & 12465,25€ \\      
        \hline
    \end{tabular}
\end{center}

\section{Which phase the project is in}

As we can see in figure \ref{fig:planning}, the project is halfway through the final game development phase.
