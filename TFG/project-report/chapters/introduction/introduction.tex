\section{Formulation of the problem}

Environmental sustainability has been a key topic for the past few years. It has been demonstrated
\cite{stateofglobalair}
that our current level of consumption and, by consequence, resource-depletion has and is having a very harmful impact on planet Earth and its biosphere. Several measures are currently being taken
\cite{climatechangepolicies}
by corporations and governments alike, but our planet may be calling for a more fast and radical change in our way of perceiving nature.

This project aims to raise questions about our current relationship with nature and our way of understanding it. In our current world of fast and efficient communication, we can no longer trust on old ways of transmitting information
\cite{teensdontread}
for the ever-increasing technologically savvy population. The world of today calls for a new form of thought-provoking projects and manifestos, it is for this reason a mobile video game will be used for this endeavor.

\section{Objectives of the project}

The objectives of this project are

\begin{itemize}
	\item Study current teaching methods for educating people about environmental sustainability.
    \item Compare and choose the method that can be best transformed into a mobile gaming experience.
    \item Implement the method as a mobile game.
    \item Analyze its impact on players by a face-to-face post-play interview.
\end{itemize}

\section{Context}

The project is developed having in mind the existence of the Serious Games movement, that is, the movement of game developers creating games that aim to not (exclusively) entertain the player, but rather make them obtain an extrinsic reward through their game-playing experience.

Some examples of serious games are flight simulators, medical surgery practice games or the Animal Equality virtual slaughterhouse experience.
\footnote{http://ianimal360.com/}

This project is a game and it is being developed under the category of a Serious Game and bibliography about them will be cited thorough the length of the thesis.
