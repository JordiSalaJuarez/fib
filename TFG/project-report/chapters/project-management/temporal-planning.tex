\section{Temporal planning}

\subsection{Programme}

The approximate length of the project is four months and three weeks, from 
February 19th until june 29th.

It is important to note that the duration of the project is approximated
and that it can be affected by the chosen project management methodology
and several other seatbacks.

\subsection{Project phases}

The project consists of several phases of development, each with their
different parts.

\subsubsection{Initial milestone}\label{initial}

The phase this section belongs to. During this phase, several aspects of the project will be thought out to help a better and more effective development of the project, such as the project's sustainability rating, its scope, its budget, etc. Work on this phase can begin as early as the requirements for it are laid out, because it does not depend on any other phase of the project to be finished.

\subsubsection{Prototype development}

This phase consists of three parts:

\begin{itemize}
    \item Research
    \item Prototype development
    \item Prototype testing \& final prototype choice 
\end{itemize}

Each one of them being dependant on the one above it. Research can begin as early as wanted, as it does not depend on any other task.

\subsubsection{Final game development}

Once prototype development is finished, the final game's development can begin. It is necessary for a final prototype to be finished before further expansion of it.

The development of the final game from the final prototype will span for four entire weeks. It is a tight deadline, although a lot of code and work will be recycled from the prototype.

\subsubsection{Player interviewing}\label{playerint}

Interviews will be done in one week. The first days will be destined to find players and the rest to actually do the interviews, either being in person or online. For this phase to begin, the final game has to be both finished and stable to ensue a correct development of this phase. 

\subsubsection{Final milestone}

This phase is the last one of the project, and depends upon the finishing of both phase \ref{initial} and \ref{playerint}, as the final documentation cannot be written without those other phases concluding.

\subsection{Schedule alterations}

It is probable that some alterations have to be applied to the project's schedule. Given the use of an Agile methodology, deadline modifications are to be expected, although the use of short sprints can help the development to be adapted to run at a higher pace if falling behind schedule.

There is a variety of seatbacks that can happen during the project's development.
It could happen that the researcher does not find a sufficient ammount of information or
cannot extract the needed information from the scientific papers in time for the project
to begin development. In this case, help will be needed from the thesis supervisor for
gathering information and recommending investigation lines. If the research finally yields
a poor result, the developer will not start until a needed minimum of theory is
available to work with. This may hinder the quality of the final game, but because
the development methodology used is incremental, it is sure that a playable game
will be produced nontheless, although with a lesser number of the expected features.

Another problem that may happen is that the game development takes too long
of a time to be developed, either because of several bugs in the code or because
extensibilty issues. This risk is mitigated by the fact that a very particular
coding architecture will be used: \gls{AMVCC}. This architecture is used
to handle large and complex codebases in Unity. Also, using an Agile methodology
will greatly help overcome any issue with bugs and non-functioning code, as
it offers great flexibility and adaptability to the development team.

At the end of the project, a review will be made detailing to what extent the original plan has been followed thanks to the reports sent to the thesis supervisor.

\subsection{Gantt diagram}

Below there can be observed a Gantt diagram depicting the duration the previously mentioned phases (and of its corresponding parts).

On the leftmost column it can be observed the phases' name, and on the rightmost one the hours each one takes to be completed. The topmost row depicts the week of development the work will take place in. Arrows indicate task dependance.

The colors on the cells represent the intensity of the work given a current time and a task. Yellow indicates medium intensity, while green and dark green depict high and very high intensity.

\adjincludegraphics[width=\textwidth,trim={0 {.5\height} 0 0},clip]{figures/timetable}

\subsection{Approximate duration}

A condensed version of the information on the duration of each phase presented on the Gantt diagram can be observed in the table below:

\begin{center}
    \begin{tabular}{ |c|c| } 
        \hline
        \textbf{Phase} & \textbf{Hours} \\
        \hline
        \hline
        Initial milestone & 70 \\ 
        \hline
        Prototype development & 100 \\ 
        \hline
        Final game development & 80 \\ 
        \hline
        Player interviewing & 40 \\ 
        \hline
        Final milestone & 40 \\ 
        \hline
        \hline
        \textbf{Total} & 330 \\ 
        \hline
    \end{tabular}
\end{center}

\subsection{Resources}

Several tools and resources will be used through the duration of this project:

\subsubsection{Hardware}

The hardware used for this project will be:

\begin{itemize}
    \item Smartphone OnePlus A5010 5T 
    \item Laptop computer Lenovo NB ideapad 110-15ISK
\end{itemize}

\subsubsection{Software}

The software tools that'll be used will be, amongst others:

\begin{itemize}
    \item Unity 3d
    \item LaTeX
    \item Trello
    \item Google Drive
    \item itch.io
    \item Ubuntu 16
    \item Windows 10
    \item Git / Github
    \item Android
\end{itemize}

