\section{Introduction} \label{sec:introduction}

% This section will be used to create a mental image on the reader's mind about the general purpose of D, and illustrate this with examples. This will make the reading much easier for her, and augment its comprehension

D, as its very name indicates, is intended to be "the next C", or, as many of its users say "the language C should have been".

Through this article the reader will see that this language is, indeed, very similar to C in many of its aspects, including its syntax, which is apparent in the code snippet in figure \ref{fig:d_syntax_code_example}.

\begin{figure}
  \caption{An example of D's syntax}
  \label{fig:d_syntax_code_example}
  \begin{lstlisting}
    import std.stdio;

    int function(char c) fp;

    void main() {
      fp = function int(char c) {
        return 6;
      };

      writefln(fp('A'));
    }
  \end{lstlisting}
\end{figure}

This language, as opposed to its more verbose and clumsy father, has a wide range of native functionality that yield a more condensed and safe code, while being comparable in speed to pure ANSI-C code.

With all of this in mind, let us proceed to the rest of the article, at the end of which the reader will have a solid understanding of the basics of this language, and will be ready to delve into D programming with confidence and ease.
