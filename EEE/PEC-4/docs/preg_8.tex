\section{Fes l’anàlisi del lloc de treball d’un teleoperador seguint el model de la pàg. 105 del llibre}

\begin{description}
  \item[Nom del lloc]
  \hfill \\
  Teleoperador
  \item[Lloc del que depèn jeràrquicament]
  \hfill \\
  Directora de RRHH, Director de CPiL (\textit{project manager} de l'obrador).
  \item[Descripció general del lloc]
  \hfill \\
  El teleoperador és responsable de rebre comandes d'entrega a domicili de diversos clients i de introduïr-les al sistema informàtic per a que es realitzin a posteriori. És també responsabilitat seva fer que aquest procés sigui el més agradable i senzill per als clients.
  \item[Descripció de tasques i funcions]
  \hfill
  \begin{enumerate}
    \item Rebre les comandes via telefònica per part dels clients
    \item Introduïr les comandes al sistema informàtic
    \item Garantir que el procés de rebuda de comandes sigui el més satisfactori pel client com sigui possible
  \end{enumerate}
  \item[Formació acadèmica]
  \hfill \\
  Educació Secundària Obligatòria
  \item[Formació no acadèmica necessària]
  \hfill \\
  Coneixement mitjà del S.O. Windows
  \item[Experiència laboral prèvia]
  \hfill \\
  Cap
  \item[Coneixements necessaris]
  \hfill
  \begin{itemize}
    \item Assertivitat i amabilitat
    \item Atenció i precisió en la presa de notes
    \item Bona ortografía i pronúncia
  \end{itemize}
  \item[Coneixements i habilitats]
  \hfill
  \begin{itemize}
    \item Bona comunicació oral i escrita
    \item Bones relacions interpersonals
    \item Energia i disciplina
  \end{itemize}
\end{description}
