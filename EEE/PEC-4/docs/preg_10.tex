\section{Quin lloc de treball creus que ha de tenir més retribució: la direcció de CPiL o la de RRHH?}

He formalitzat la meva resposta en format taula, on a cada criteri se li assigna un número de 0 a 1 en funció del seu grau d'importància  cada càrrec, sent 1 el máxim i 0 el mínim. El lloc de treball que hauria de tindre més retribució econòmica és aquell que té una suma total dels criteris més gran que l'altre.

Algunes aclaracions sobre la meva interpretació de alguns dels criteris escollits:

\begin{description}
  \item[Maginitud econòmica]
  \hfill \\
  Repercussió que tenen les accions d'un individu sobre l'empresa a nivell econòmic
  \item[Impacte en la organització]
  \hfill \\
  Repercussió que tenen les accions d'un individu sobre l'empresa a nivell organitzatiu i/o estructural
\end{description}

\begin{tabular}{p{8cm} | p{2cm} | p{2cm}}
  \textbf{Criteris} & \textbf{Dir. CPiL} & \textbf{Dir. RRHH} \\
  \hline
  \hline
  \multicolumn{3}{l}{\textbf{Competències}} \\
  \hline
  Competència tècnica & 0.8 & 0.2 \\
  Competència gerencial & 0.8 & 0.8 \\
  Competència d'interacció humana & 0.6 & 1 \\
  \hline
  \multicolumn{3}{l}{\textbf{Resolució de problemes}} \\
  \hline
  Complexitat dels problemes & 0.8 & 0.5 \\
  Importància dels problemes & 0.8 & 0.9 \\
  \hline
  \multicolumn{3}{l}{\textbf{Responsabilitat}} \\
  \hline
  Quantitat de poder executiu & 0.9 & 0.9 \\
  Maginitud econòmica & 0.8 & 0.8 \\
  Impacte en la organització & 0.7 & 0.5 \\
  \hline
  \multicolumn{3}{l}{\textbf{Condicions de treball}} \\
  \hline
  Perillositat & 0 & 0.1 \\
  Periodicitat & 0.9 & 0.9 \\
  \hline
  \hline
  \textbf{Total} (mitjana) & 0.71 & 0.66 \\
  \hline
\end{tabular}
\\ \\
Podem concloure, llavors que el lloc de Director de Compres, Producció i Logística mereix una retribució econòmica major basant-nos en aquests criteris.
