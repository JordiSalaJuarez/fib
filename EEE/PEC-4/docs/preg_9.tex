\section{Cóm faries el reclutament dels venedors de les botigues? Quines fonts de reclutament empraríeu? I pels encarregats de tenda?}

Pels venedors de les botigues utilitzaria fonts de reclutament externes, ja que no tindria sentit emprar reclutament inter per una \textit{entry level position} (posició de responsabilitat reduïda).

Per als encarregats de la tenda utilitzaria reclutament intern i escolliria d'entre els millors venedors de les botigues, ja que fomentaria l'esperit de competició i a més a més donaria a la empresa una imatge meritocràtica que motivaria als empleats a créixer professionalment.

\subsection{La figura \ref{fig:zara} és d’un formulari agafat a una botiga de l’empresa Zara. Us semblaria adient aquest tipus de canal per els venedors? Perquè? Perquè ho fa Zara?}

Sí. Em sembla adient, ja que, encara que és reductivista, proporciona un marc molt simple per encapsular les qualitats que més interessen a \QnF sobre els seus venedors, i permetent així processar un gran volum de sol·licituts en un temps molt reduït comparat amb el processament de CV's clàssics.

Zara ho fa perquè, com que és una marca reconeguda arreu del món i amb botigues repartides per tot el país, poden aprofitar aquesta enorme xarxa de centres de venda com a centres de reclutament. Aquests currículums normalment estàn al cantó de la caixa, i qualsevol persona que hagi comprat a Zara pot potencialment agafar el currículum i plenar-lo per demanar un treball. Aixó, juntament amb el que hem esmenat abans, genera un avantatge considerable respecte a vies més tradicionals de reclutament.

\begin{figure}[!tbp]
  \centering
  \subfloat{\includegraphics[width=0.5\textwidth]{zara.jpg}}
  \hfill
  \subfloat{\includegraphics[width=0.5\textwidth]{zara2.jpg}}
  \caption{Formulari de reclutament d'una botiga de Zara}
  \label{fig:zara}
\end{figure}
