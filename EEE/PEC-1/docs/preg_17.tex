\section {
  Sovint es parla molt de Xina, com “el gegant asiàtic”, ja que certament
  té una població de més de mil milions d’habitants, 
  i produeix moltíssims bens.
  Al mateix temps però, s’oblida l’Índia, 
  un país amb una quantitat de població similar a la
  de la Xina, i que en comptes de produir tants bens materials com la Xina,
  es dedica a “produir” bens immaterials (coneixement). 
  Ara per ara, l’Índia ja és un país líder en el
  sector de la informàtica i la programació. 
  Creus que l’Índia és una amenaça econòmica pels països occidentals? 
  O és una oportunitat? Defensa els teus arguments.
}

Personalment crec que la Índia podria arribar a ser una amenaça important
si les seves infraestructures es desenvolupen al nivell de les d'un país
del primer món.

Qualssevol pot programar, i si l'Índia es comença a associar a un segell
de qualitat i professionalitat informàtiques, és molt possible que la 
majoria de treball que es fa a Occident acabi sent realitzar remotament
per equips indis (el que ja està succeïnt, encara que a una escala 
controlada). Això generaria una quantitat d'atur impressionant, i 
esfonsaria moltes empreses de desenvolupament al no poder competir amb
els costos de contratació de la india.
