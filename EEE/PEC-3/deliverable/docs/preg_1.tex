\section{De què dependrà el valor del ``pankon''?}

El valor d'una moneda, en general, depèn de varis factors 
\cite{currency_value}:

\begin{description}
  \item[Inflació] Països amb una inflació petita veuen que la seva moneda
  augmenta de valor.
  \item[Tipus de canvi] Països amb un tipus de canvi superior tenen monedes
  millor valorades.
  \item[Estabilitat econòmica] En general, un país amb una bona estabilitat
  econòmica és percebut com un bon mercat per al negoci, el que fa que
  el valor de la seva moneda augmenti amb el temps.
  \item[Inversió estrangera] A més inversió estrangera, millor serà el 
  posicionament de la moneda del país que la rep.
  \item[Importacions/exportacions] Si un país importa més del que exporta
  la seva moneda es veu deprecada. En el cas contrari, la moneda augmenta
  de valor.
\end{description}

El cas del pankon no és diferent, i podem utilitzar els mateixos principis
que serveixen pel valor general de la moneda. En concret podríem assenyalar
que, donada la situació econòmica de Pankilàndia, és molt probable que el
pankon experimenti una devallada de valor.

