\subsection{Quines reflexions fa el premi Nobel d’economia Paul Krugman
sobre l’afirmació que fan Carmen Reinhart i Kennneth Rogoff en el seu 
article ``Growth in a time of debt'' (la depresión del Excel) sobre que
una vegada el deute públic d’un país ha superat el
90\% del seu PIB el seu creixement cau en picat?}

El senyor Krugman afirma, al seu article 
que les troballes dels investigadors Reinhart i Rogoff són donades per un
error de codificació d'una fulla excel. La premissa que defensen és la de
que en un país en el qual la relacio de deute amb el seu PIB és superior al
0,9, la economia d'aquest es veu afectada molt negativament i entra en 
colapse.

El problema és que aquesta afirmació no ha set validada per cap altre 
economista ja que no han aconseguit replicar-la usant dades reals. Quan
els investigadors van alliberar la fulla excel sobre la cual van fer el 
estudi, altres economistes es van adonar de que les dades estaven mal
tractades i hi havia un error de codi al excel.

El problema important, subratlla Krugman, és la rapidesa amb la que es van
acceptar les afirmacions d'aquesta gent, sense tenir en compte, óbviament,
la seva rigurositat o el seu nivell de credibilitat entre la comunitat 
economista. Krugman senyala que aquesta facilitat per acceptar les 
afirmacions ve de la por irracional dels països al deute, cosa que Krugman
assenyala com a nociu.
