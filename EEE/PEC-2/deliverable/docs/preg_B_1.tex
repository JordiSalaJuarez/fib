\subsection{Posa 5 pegues a utilitzar el PIB per càpita (PPP) com un índex
de desenvolupament dels països}

\begin{enumerate}

  \item El PIB deixa de banda
  aspectes relacionats amb variables fixes de la economia, ja
  que al ser una variable flux recompensa la depredació de recursos naturals
  en poc temps i no té en compte la pèrdua d'aquests recursos, per exemple.
  
  \item Des del punt de vista de comptabilitat nacional tindran 
  igual aportació al PIB la despesa en un
  milió d’euros en serveis educatius que en armament militar, no obstant, 
  resulta obvi que la contribució de cada
  una d’aquestes despeses al benestar agregat de la societat és ben diferent. 
  
  \item El PIB exclou totes aquelles activitats que es realitzen al marge del 
  circuit mercantil,
  encara que satisfacin necessitats altament valorades. Un cas exemplar
  en aquest àmbit és la feina lligada a la maternitat i l’atenció als fills.
  
  \item El PIB és incapaç de valorar adequadament l’aportació d’aquelles activitats
  no registrades legalment a la generació de renda d’un país. 
  Aquest aspecte resulta transcendent en el cas dels països en 
  desenvolupament.
  
  \item El PIB és un càlcul que, degut a causes com la del anterior punt, és 
  imprecís de calcular. 
  En els països desenvolupats son freqüents errors propers al 10 per 100 en 
  les estimacions, però en els països en desenvolupament aquests errors 
  s’amplifiquen, fins a arribar a valors propers al 20 por 100 en els 
  pitjors casos.

\end{enumerate}
