\section{Munich sembla no veure convenient deslocalitzar la producció a 
països amb baixos costos laborals.
Perquè? Què té a veure això amb el just in time?}

Els avantatges de la deslocalització són obvis. Però té un gran inconvenient:
el temps i la producció al detall. En altres paraules: la deslocalització de
la producció en mercats més competitius és útil per generar grans quantitats
de producte igual, amb un temps d'enviament de vàries setmanes.

El problema és que Munich ofereix un servei molt concret per als seus clients:
"Munich your way", el que permet escollir entre milions de permutacions de
color-decoracions per a les teves sabatilles al teu gust, i rebre-les en menys
de dues setmanes a casa teva. De la fàbrica al consumidor en temps rècord. El
problema és que no es pot fer això quan la teva producció es troba tan 
deslocalitzada. Per això Munich no ho fa.

La no-deslocalització de la producció és clau per a que una empresa pugui
implementar el JIT manufacturing. La fluidesa de la producció i la reducció
de la latència del feedback entre la empresa i la manufactura són peces clau
del JIT. La deslocalització entra en conflicte directe amb aquests principis.
