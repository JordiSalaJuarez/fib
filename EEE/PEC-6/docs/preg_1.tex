\section{ En quant a d’investigació i anàlisi de mercats, respon a les següents 
preguntes sobre el cas Munich }

\subsection{ Qui son els seus clients? }

Munich es dedica al mateix temps a la venda a l'engròs i al detall, motiu pel
que els seus cients són, al mateix temps:

\begin{itemize}
  \item Botigues de moda i/o de roba esportiva
  \item Persones de totes les edats i sexes residents a europa.
\end{itemize}

\subsection{ Quin producte o servei està venent? }

Calçat i roba esportiva, i també accessoris i roba de moda.

\subsection{ Quin es el seu mercat objectiu? }

Persones de totes les edats amb un poder adquisitiu mitjà alt que els hi agradi
la personalizació, la moda streetwear o que practiquin algun esport. El mercat
objectiu, geogràficament parlat, de moment està limitat a Europa.

\subsection{ Qui son els seus competidors i què fan? }

Els seus competidors són totes les empreses internacionals de roba esportiva i
alguns falsificadors que operen a nivell nacional i internacional.

Les marques esportives estàn centrades en aconseguir que esportistes famosos
reconeguin les seves marques, i estàn obessionats amb la imatge corporativa.

Els falsificadors de Munich intenten vendre falsificacions dels productes de la
empresa de menor qualitat al mateix preu que el de un d'original.
Això fa que la marca Munich es vegi desprestigiada.
