\section{Como las herramientas de trabajo colaborativo mejoran la sostenibilidad de un proyecto TIC}

Las herramientas de trabajo colaborativo ya forman parte de las organizaciones modernas, y se está extendiendo a un ritmo vertiginoso gracias a la incorporación de nuevas tecnologías accesibles y potentes como el smartphone a la vida diaria de la oficina. En 2018, el 50\%\cite{adam} de la coordinación entre equipos sucederá vía aplicaciones móvil de colaboración. Hay y ha habido siempre una necesidad imperativa dentro de las organizaciones de colaborar y cooperar entre ellas de manera más efectiva, aunque los últimos años se puede observar un aumento en el número de agentes externos necesarios para desempeñar cualquier trabajo en una empresa, hasta llegar a que el 71\%\cite{ceb} de de organizaciones colaboran regularmente con agentes externos a esta.

\subsubsection{Económico}

Los beneficios económicos del uso de herramientas colaborativas en el lugar de trabajo son extensos y han sido demostrados efectivos para mejorar la eficiencia y la comunicación entre empleados.

Las HTC permiten que la bolsa de trabajo de personas a emplear sea mucho más grande gracias a la posibilidad del trabajo remoto. Ya han pasado los días de trabajo presencial obligatorio en las oficinas, y las empresas no tienen por qué restringirse a su ámbito local para la búsqueda de talentos. El trabajo remoto no existiría o por lo menos no estaría tan extendido sin las existencia de HTC’s accesibles para la mayoría de usuarios en el mundo.

Las herramientas de videoconferencia son usadas en las empresas para generar un mayor sentido de pertenencia a la organización y que los trabajadores se sientan más parte del equipo. También permite a los responsables de las organizaciones reunirse “cara a cara”   con personal externo o trabajadores remotos sin necesitar perder tiempo en transporte. A través del uso de estas herramientas la toma de decisiones importantes se acelera y se mejora su calidad final, al poseer un método de telecomunicación tan rápido y directo. Numerosos estudios respaldan estas afirmaciones, teniendo en cuenta que los resultados de estos señalan que un 87\%\cite{zeus} de los trabajadores remotos se sienten más conectados a su equipo y trabajo cuando usan tecnologías de videoconferencia o que un 75\%\cite{zeus} de usuarios que usan extensivamente herramientas de videoconferencia dicen que estas mejoran la colaboración y la productividad en las empresas.

Otra herramienta de comunicación, la mensajería instantánea, también plantean generosos beneficios a la hora de mejorar la sostenibilidad económica de un proyecto. Ventajas como la posibilidad de conservar las conversaciones realizadas vía texto entre miembros del equipo o la posibilidad de forjar relaciones semi formales con proveedores u otros agentes externos gracias a estas herramientas genera un ambiente de trabajo estable y relajado, sin la necesidad de que la comunicación interpersonal no sea rígida y burocrática, abriendo las puertas a una manera de comunicarse más flexible, tanto internamente como fuera de la organización.

Las herramientas de trabajo colaborativo permiten, sumadas al uso de dispositivos móviles, la posibilidad de que las empleados se sientan siempre conectadas a su lugar de trabajo. Este aspecto de las HTC puede ser negativo de cara a la salud mental de los empleados y la división entre el trabajo y la vida en familia de estos mismos. Usados, pero, de una manera moderada, pueden ayudar a que las empleadas puedan comunicarse entre ellas aunque la oficina esté cerrada, abriendo la puerta a la posibilidad de colaborar con trabajadoras fuera del horario de oficina, aumentando así la productividad de la empresa y el sentimiento de apego de los empleados al producto.

Fuera del ámbito del lugar de trabajo, las HTC también se encuentran presentes, y su presencia es muy necesaria para mejorar la relación entre los diversos agentes que intervienen en el pipeline productivo de una empresa. Tanto es así que de un 31\% a un 33\% de organizaciones están actualmente desarrollando estrategias para incrementar la comunicación entre empresa y proveedores a través de HTC\cite{forrester}, por no hablar de que un 53\%\cite{ibm} de las directoras de empresas tan solo aúnan esfuerzos con otras empresas para entablar relaciones de colaboración innovativa, tales como las que se han visto anteriormente en la sección \ref{sec:impact_on_society}. Estos datos no son una exageración, ya que, según estudios, un 65\%\cite{ceb} de las empleadas en organizaciones requieren comunicarse con agentes externos para desenvolver su trabajo, lo que justifica el interés que las empresas demuestran sobre las HTC en el aspecto económico.

\subsubsection{Social}

Las HTC presentan claros beneficios a la sostenibilidad empresarial si las estudiamos desde la perspectiva social. En 2014, un 25\%\cite{frost} de las empleadas en organizaciones trabajaba desde casa, y no podemos hacer otra cosa que suponer que el porcentaje ha aumentado desde entonces. Las HTC posibilitan que este trabajo remoto sea accesible para estas personas, y abren incontables posibilidades de trabajo para gente sin la capacidad adquisitiva suficiente para costearse un vehículo privado o transporte público, para personas con discapacidades motrices o para las que padecen enfermedades mentales.

Siguiendo con la idea de las HTC posibilitando el trabajo remoto, los equipos dentro de organizaciones son, hoy en día, potencialmente multiculturales y más abiertos a la posibilidad de trabajar con personas extranjeras y de otras ideologías. El lugar de trabajo es un espacio donde se comparten ideas y se forjan relaciones, y a través de posibilitar el trabajo remoto y introducir HTCs de naturaleza más laxa como la mensajería instantánea, posibilita la forja de relaciones y intercambios de ideas entre compañeras de trabajo donde una o ambas trabajan remotamente, generando una sociedad más inclusiva a través de un lugar de trabajo más abierto y conectado globalmente.

Otra de las ventajas sociales que presenta el trabajo remoto potenciado por las HTC es el de que los precios de las viviendas en los centros de población tiende a descender drásticamente en función del porcentaje de personas que trabajan remotamente en las empresas situadas en la zona\cite{adrianne}.

\subsubsection{Medioambiental}

Las HTC aumentan la eficiencia de las empresas, sobretodo en las que emplean a trabajadoras remotas. Un aumento de este tipo permite reducir redundancias en la organización, lo que a su vez reduce el consumo energético de estas. Donde más destaca su impacto ambiental es en la colaboración interempresarial, donde varias organizaciones valoran sus intereses comunes en respecto a la explotación, manipulación y desecho de recursos naturales y autorregularse para alcanzar acuerdos que beneficien a todos los agentes implicados.

En los casos como el de Dairy Management Inc, donde una serie de empresas ganaderas se pusieron de acuerdo en 2007 al fundar una organización bajo la cual todas las agentes implicadas convergen para buscar soluciones a los problemas de sostenibilidad ambiental que presenta la industria ganadera. La organización tiene como objetivo el de reducir las emisiones de CO2 en un 25\% en 2020. Es una iniciativa de peso en la que participan empresas que acumulan hasta un 75\% de la producción total de leche en los EEUU\cite{ram}.

Las HTC, al facilitar el trabajo remoto, reducen el uso de vehículos privados al anular la necesidad de transporte de las empleadas que trabajan desde su domicilio. En el caso de Xerox, el emplear 1000 trabajadoras para que desempeñen sus tareas remotamente en 2014 supuso un ahorro de 17’5 millones de litros de gasolina al año y una reducción en las emisiones de casi 41000 toneladas métricas anuales\cite{adrianne}.

En general, una empresa que use más HTC TIC gastará menos papel, como en el caso de Aetna, gran impulsora del trabajo remoto colaborativo, donde el 43\% de sus trabajadoras trabajan remotamente\cite{adrianne}. La adopción de estas herramientas para la comunicación interpersonal dentro de una organización genera menos burocracia y menos necesidad de usar papel para transmitir información. Teniendo en cuenta que para producir una sola hoja de papel necesitamos más de 10 litros de agua\cite{megan}, es vital que las empresas adopten estas iniciativas TIC para reducir su huella ecológica.
