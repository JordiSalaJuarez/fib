\section{Herramientas de trabajo colaborativo en las TIC}

\subsection{Que es una herramienta de trabajo colaborativo}

Una herramienta de trabajo colaborativo es una herramienta que ayuda
a colectivos de personas a colaborar con la finalidad de alcanzar
objetivos comunes de manera más eficiente.

Las herramientas de trabajo colaborativo pueden ser no-tecnológicas
como el lápiz y el papel o estar totalmente basadas en sistemas
informáticos como el correo electrónico.

\subsection{Tipos de herramientas de trabajo colaborativo en las TIC} \label{ref:tool_types}

Las herramientas de trabajo colaborativo, en el campo de las TIC, se pueden
clasificar según su función (a qué cometido han sido destinadas) o según su
dimensión (como son utilizadas por los usuarios)\cite{ashley}.

\subsubsection{Tipos de herramientas según su función}

\paragraph{Comunicación}

Son las herramientas que se basa en facilitar la transmisión de información entre las diversas participantes de una organización. Las herramientas más comunes que nos podemos encontrar de este tipo son, por ejemplo, el correo electrónico, la mensajería instantánea, los foros online, las enciclopedias electrónicas, etc. 

\paragraph{Coordinación}

Las herramientas que ayudan a la coordinación son servicios que ayudan a que el trabajo y los objetivos de las personas dentro de una organización se alineen y solidifiquen en algo tangible. Son herramientas que, por ejemplo, ayudan a concretar fechas para realizar acciones grupales o para ajustar horarios, como los calendarios electrónicos, herramientas de gestión de producto, etc.

\paragraph{Colaboración}

Las herramientas de colaboración son las que ayudan a las miembros de una organización a dar forma a una idea conjuntamente o pensar sobre un tema de manera conjunta. La videollamada o las pizarras en línea son los ejemplos más comunes de este tipo de herramientas.

\subsubsection{Tipos de herramientas según su dimensión temporal}

\paragraph{Síncronas}

Herramientas que son usadas por varias personas que requieren cierto grado de simultaneidad en su uso, es decir, herramientas que funcionan correctamente si son usadas al mismo tiempo por varias personas. Ejemplos de este tipo de herramientas pueden ser la mensajería instantánea, las aplicaciones de documentos compartidos en tiempo real o las videollamadas.

\paragraph{Asíncronas}

Herramientas que no requieren que los diversos usuarios que las utilizan estén haciéndolo al mismo tiempo para que la colaboración sea efectiva. Algunos ejemplos de este tipo de herramientas podrían ser el correo electrónico,. los sistemas de control de versiones colaborativos, los calendarios grupales o los programas de gestión de proyectos.

\subsection{Ejemplos contemporáneos}

A continuación, se enuncian varios ejemplos contemporáneos de HTC en las TIC. Se categorizarán en función de los tipos de herramienta que se han enunciado en la sección \ref{ref:tool_types}.

\noindent\rule{\textwidth}{0.5pt}

{ \large \textbf{Slack} }

\noindent\rule{\textwidth}{0.5pt}


\begin{description}
    \item[Dimensión temporal] Síncrona 
    \item[Función] Planificación
    \item[Tipo] Mensajería instantánea 
    \item[Plataformas] Web, Escritorio, Móvil 
    \item[Monetización]
    \item[Descripción]
    Aplicación de mensajería orientada a los grupos de trabajo. Se puede dividir en canales para mejorar la localización de información clave y separar temas dentro de un mismo equipo.
\end{description}

\noindent\rule{\textwidth}{0.5pt}

{ \large \textbf{Google Keep} }

\noindent\rule{\textwidth}{0.5pt}

\begin{description}
    \item[Dimensión temporal] Asíncrona 
    \item[Función] Comunicación
    \item[Tipo] Documentos en línea
    \item[Plataformas] Web, Escritorio, Móvil 
    \item[Monetización]
    \item[Descripción]
    Usada para compartir ideas entre miembros de una misma organización. Google intenta emular el uso de los post-its con esta herramienta, haciendo que sean fácilmente compartibles y sencillos de dividir por categorías mediante una interfaz simple y visual.
\end{description}

\noindent\rule{\textwidth}{0.5pt}

{ \large \textbf{Google Drive} }

\noindent\rule{\textwidth}{0.5pt}

\begin{description}
    \item[Dimensión temporal] Síncrona
    \item[Función] Colaboración
    \item[Tipo] Documentos en línea
    \item[Plataformas] Web, Escritorio, Móvil 
    \item[Monetización]
    \item[Descripción]
    Compartir y editar documentos en la nube nunca ha sido tan fácil. Google Drive ayuda a las organizaciones a mantener todos sus archivos en la nube y permite a las empleadas editarlos y/o compartir ideas y comentarios sobre ellos.
\end{description}

\noindent\rule{\textwidth}{0.5pt}

{ \large \textbf{Trello} }

\noindent\rule{\textwidth}{0.5pt}

\begin{description}
    \item[Dimensión temporal] Asíncrona
    \item[Función] Planificación
    \item[Tipo] Gestión de proyectos
    \item[Plataformas] Web, Escritorio, Móvil 
    \item[Monetización] Gratis
    \item[Descripción]
    Pizarra virtual para organizar un equipo basado sobre todo en metodologías de trabajo ágil. Interfaz sencilla y accesible.
\end{description}

\noindent\rule{\textwidth}{0.5pt}

{ \large \textbf{appear.in} }

\noindent\rule{\textwidth}{0.5pt}

\begin{description}
    \item[Dimensión temporal] Síncrona
    \item[Función] Comunicación
    \item[Tipo] Videollamada
    \item[Plataformas] Web
    \item[Monetización] Gratis
    \item[Descripción]
    Aplicación web que soporta videollamadas grupales de hasta 8 personas.
\end{description}

\noindent\rule{\textwidth}{0.5pt}

{ \large \textbf{RedPen} }

\noindent\rule{\textwidth}{0.5pt}

\begin{description}
    \item[Dimensión temporal] Síncrona
    \item[Función] Comunicación
    \item[Tipo] Pizarra virtual
    \item[Plataformas] Web
    \item[Monetización] 20\$ - 90\$ al mes
    \item[Descripción]
    Pensada para diseñadores, RedPen permite a estos enseñar sus productos en tiempo real a sus clientes o agentes internos de la organización que deben valorar su trabajo. A través de comentarios que se pueden realizar directamente sobre una imagen propuesta. Cabe destacar que la aplicación guarda un histórico de las versiones presentadas para que el usuario pueda recuperarlas si lo desea.
\end{description}

\noindent\rule{\textwidth}{0.5pt}

{ \large \textbf{Asana} }

\noindent\rule{\textwidth}{0.5pt}

\begin{description}
    \item[Dimensión temporal] Asíncrona
    \item[Función] Coordinación
    \item[Tipo] Gestión de proyectos
    \item[Plataformas] Web / Android
    \item[Monetización] Gratis (Prueba gratis)
    \item[Descripción]
    Asana es una aplicación que permite a varias miembros de la organización asignarse tareas entre ellas, gestionar fechas de entregas y comunicarse entre ellas vía salas de mensajería instantánea por proyecto.
\end{description}

\noindent\rule{\textwidth}{0.5pt}

{ \large \textbf{Chanty} }

\noindent\rule{\textwidth}{0.5pt}

\begin{description}
    \item[Dimensión temporal] Asíncrona
    \item[Función] Coordinación
    \item[Tipo] Gestión de proyectos
    \item[Plataformas] ?
    \item[Monetización] ?
    \item[Descripción]
    Se trata de una herramienta que usa técnicas de IA para gestionar organizaciones. No se sabe mucho de como es, pero es una de las novedades que se podrán ver en 2018.
\end{description}

\noindent\rule{\textwidth}{0.5pt}

{ \large \textbf{Salesforce} }

\noindent\rule{\textwidth}{0.5pt}

\begin{description}
    \item[Dimensión temporal] Asíncrona
    \item[Función] Coordinación
    \item[Tipo] CRM (Gestor de Relaciones con Clientes)
    \item[Plataformas] Web, Escritorio, Móvil
    \item[Monetización] Pago
    \item[Descripción]
    El CRM más famoso de todos. Las CRMs son herramientas que ayudan a almacenar y gestionar datos sobre clientes y los negocios hechos con ellos. Imprescindible para empresas de ventas a llamada fría.
\end{description}

\noindent\rule{\textwidth}{0.5pt}

{ \large \textbf{Cozi} }

\noindent\rule{\textwidth}{0.5pt}

\begin{description}
    \item[Dimensión temporal] Asíncrona
    \item[Función] Coordinación
    \item[Tipo] Gestión de proyectos
    \item[Plataformas] Web, Escritorio, Móvil
    \item[Monetización] Gratis
    \item[Descripción]
    Una excepción en el mercado de las HTC. Herramienta de gestión de proyectos orientada a familias. Cozi permite organizar a tu familia como si fuese una startup. Que eso sea positivo o no no se va a mencionar en este trabajo. 
\end{description}

\noindent\rule{\textwidth}{0.5pt}

{ \large \textbf{Basecamp} }

\noindent\rule{\textwidth}{0.5pt}

\begin{description}
    \item[Dimensión temporal] Asíncrona
    \item[Función] Coordinación
    \item[Tipo] Gestión de proyectos
    \item[Plataformas] Web, Escritorio, Móvil
    \item[Monetización] Prueba gratis
    \item[Descripción]
    La madre de todas las herramientas colaborativas. Desde 2004, Basecamp presenta una solución unificada a los problemas más comunes de la organización empresarial. Permite programar recordatorios para que las miembros de una organización escriban sus que han hecho durante el día o que planean hacer el día siguiente, así como gestionar las tareas del equipo.  
\end{description}