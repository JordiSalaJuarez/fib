\section{Como el trabajo colaborativo impacta en la sociedad}\label{sec:impact_on_society}

Desde las pirámides de Giza hasta la el Acuerdo de París, los grandes proyectos del hombre han surgido gracias a la colaboración entre individuos u organizaciones cuyos intereses se alinean para alcanzar un objetivo común.

A través de la colaboración, las sociedades han ido incrementando su complejidad gradualmente hasta alcanzar su nivel actual, donde transbordadores espaciales surcan las proximidades de nuestro sistema solar y donde una gran parte de la humanidad está intercomunicada entre ella mediante la Internet.

El trabajo colaborativo impacta a las sociedades de diversas maneras en función de su estructura y jerarquías, pero podríamos afirmar que todas las civilizaciones modernas empezaron cuando la colaboración interindividual empezó a proliferar. Los primeros reinos de la historia antigua se forjaron gracias al hecho de que sus habitantes empezaron a colaborar para crear y mantener plantaciones de vegetales, fundando así el pilar de todas las civilizaciones posteriores.

\paragraph{En Sociedades Actuales}

Hoy en día, en las sociedades occidentales donde la iniciativa privada rige una gran parte de nuestras vidas, una de las formas de trabajo colaborativo que más consigue impactar nuestras sociedades es aquél en el que están implicadas varias empresas privadas entre ellas, como podría ser el caso de la Latin American Water Funds Partnership \cite{ram}.

La LAWFP es una iniciativa PES (Power and Energy Solutions) que cobra a las empresas por el mantenimiento de los recursos naturales que están explotando. Por ejemplo, si una empresa forestal tala 500 árboles de un bosque protegido por la LAWFP, la empresa deberá reembolsar una cantidad equivalente al coste económico de replantar esos árboles.

Esta organización fue fundada en 2011 y cuenta con miembros como Coca Cola, proveedores de agua potable y ONG’s como The Nature Conservancy y The Global Environment Facility y el banco Interamericano de Desarrollo. Los participantes en esta iniciativa aportan capital para pagar a granjeros, terratenientes y gobiernos locales que operan río arriba para la conservación y mantenimiento de las fuentes de agua. Acciones como la reforestación, la mejora de técnicas ganaderas, implantación de técnicas sostenibles para la agricultura y esfuerzos por reducir la erosión del terreno son algunos ejemplos de las prácticas que estas organizaciones esperan de los diversos agentes que operan río arriba.

La LAWFP hoy en día se financia con 32 fondos locales, sumando un total de 27 millones de dólares. Está presente en varios países de América Latina y busca maximizar el beneficio de las empresas a través de mejorar la calidad del agua de las fuentes naturales de la región. Se espera que la iniciativa beneficie directamente a más de 50 millones de personas y mejore 3 millones de hectáreas de ecosistemas naturales.

