\section{Que es el trabajo colaborativo comunitario}

El trabajo colaborativo comunitario es mucho más (y más difícil) que trabajar en equipo \cite{rob}. Los integrantes del colectivo deben adaptarse a grupos flexibles de trabajo, donde tendrán que responder a objetivos comunes y contribuir en el aprendizaje de los demás integrantes. 

El objetivo último del trabajo en colaboración es que, además de trabajar en equipo, se haga de manera colaborativa. Aunque el concepto en sí sea más exigente y ambicioso, a cambio se consiguen equipos más eficaces y productivos. En el trabajo colaborativo los departamentos deben estar configurados de manera flexible para que la comunicación entre los integrantes sea fluida y puedan adaptarse según las necesidades de los diferentes proyectos. Así pues, se crean espacios seguros donde cada integrante puede aportar su punto de vista. 

La introducción en el mundo laboral de la tecnología ha evolucionado notoriamente en las últimas décadas \cite{canal}, incluyendo métodos de aprendizaje, donde ya no se valora las metas individuales tradicionales, sino que se tiene en cuenta el esfuerzo colectivo a través de herramientas como e-learning y el aprendizaje colaborativo.
