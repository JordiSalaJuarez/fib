\section{Conclusiones}

Se ha visto como el trabajo colaborativo no solo es beneficial para la sociedad en algunos casos, si no que es una parte intrínseca de ella. Se han presentado las diversas tipologías de herramientas de trabajo colaborativo en las TIC y se han enunciado varios ejemplos dividiéndolos en las tipologías enunciadas. Se ha explicado con ejemplos por qué estas herramientas son útiles para mejorar la sostenibilidad de un proyecto si se aplican correctamente.

La reflexión final que pretende dar este trabajo es que las HTC en las TIC son aplicaciones informáticas muy útiles que, bien utilizadas pueden llegar a mejorar sustancialmente la sostenibilidad de un proyecto empresarial en todos sus aspectos. El hecho de utilizar HTC’s no implicaría, pero, la mejora automática de la sostenibilidad de un proyecto, ya que un uso indebido o excesivo de las HTC pueden generar ineficiencias y redundancias en el tejido organizativo del equipo, lo que comprometería la capacidad de las empresas para mejorar su sostenibilidad gracias a ellas.