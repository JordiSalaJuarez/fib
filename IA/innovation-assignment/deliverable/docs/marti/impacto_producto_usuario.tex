\section{Impacto del producto en el usuario y en la sociedad}

The grid es una empresa que quiere sustituir la figura del diseñador de 
web por un sistema que utiliza inteligencia artificial. Por este motivo
la empresa ha tenido un fuerte impacto social. Este impacto se puede ver
desde distintos puntos de vista, des del punto de vista del diseñador,
des del punto de vista del usuario final, des del punto de vista de la
gente que quiere crear una web ...

Los impactos que tiene sobre los diseñadores web son los siguientes, por
una parte se ven amenazados ya que esta empresa quiere quitarles el 
trabajo, por ese motivo se quejan de que la ia no puede hacer reglas,
por otra parte hay otros que creen que este sistema puede crear unas
páginas con una calidad muy decente y que aparte se puede rediseñar a sí
misma para mejorar. Otros critica que al usar el js para modelar el css
hace el programa muy ineficaz.

Los impactos que tiene sobre el usuario final es una página que 
estéticamente es bastante decente y aparte ha sido barata de crear 
lo que le permitirá a la empresa que la ha creado invertir en otras
cosas que puede que le interesen más al usuario, por otra parte el
usuario tiene la desventaja que estas páginas suelen ser más lentas
cosa que  no suele ser de su agrado.

Otro punto de vista interesante para valorar el impacto es el impacto
que ha tenido sobre las empresas o personas que quieran diseñar una web
sin necesidad de muchos conocimientos. Des del punto de vista de las
personas que quieren crear una web esta herramienta les permite crear
webs sin necesidad de tener un gran conocimiento de diseño.
Por otra parte para las empresas es un gran aliado ya que pueden crear
webs con un buen diseño a un coste mucho más reducido que el que tendrían
si tuvieran que contratar a un profesional que lo hiciese por ellos.
