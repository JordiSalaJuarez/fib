\subsection{Experimento 2}

\paragraph{Enunciado}

Determinar qué estrategia de generación de la solución inicial da mejores resultados para la función heurística usada en el apartado anterior, con el escenario del apartado anterior y usando el algoritmo de Hill Climbing.

A partir de estos resultados deberéis fijar también la estrategia de generación de la solución inicial para el resto de experimentos.

\paragraph{Planteamiento}
Disponemos de dos maneras de generar el estado inicial. Haremos 10 pruebas con el algoritmo de Hill Climbing en cada uan y valoraremos cual da mejores resultados en funcion del heuristico y del coste de la solución.

\paragraph{Resulatdos}
\begin{table}[htb]
\centering
\begin{tabular}{l|r}
Heuristic & Cost  \\\hline
97033 & 61312 \\
108773 & 72673 \\
111896 & 77300 \\
116329 & 77125 \\
119721 & 89792 \\
115509 & 82020 \\
112056 & 75575 \\
108945 & 78320 \\
112639 & 75003 \\
101969 & 72040 \\

\end{tabular}
\caption{\label{tab:widgets}Estado inicial completo.}
\end{table}

\begin{table}[htb]
\centering
\begin{tabular}{l|r}
Media heuristico & Media coste  \\\hline
110487 & 76116 \\
\end{tabular}
\caption{\label{tab:widgets}Valores medios de los heuristicos y costes de solución para estado inicial completo.}
\end{table}

\begin{table}[htb]
\centering
\begin{tabular}{l|r}
Heuristico & Coste  \\\hline
97033 & 61312 \\
108773 & 72673 \\
111896 & 77300 \\
116329 & 77125 \\
119721 & 89792 \\
115509 & 82020 \\
112056 & 75575 \\
108945 & 78320 \\
112639 & 75003 \\
101969 & 72040 \\

\end{tabular}
\caption{\label{tab:widgets}Estado inicial aleatorio.}
\end{table}

\begin{table}[htb]
\centering
\begin{tabular}{l|r}
Media heuristico & Media coste \\\hline
113819.5 & 81621.5\\
\end{tabular}
\caption{\label{tab:widgets}Valores medios de los heuristicos y costes de solución para estado inicial aleatorio.}
\end{table}


\paragraph{Conclusiones} Podemos observar que tanto el valor medio del heuristico como el valor medio del coste de la solución es más bajo cuando emezamos con un estado completo (Cuadros 2 y 4).\\
Esto podría ser debido a que le damos más datos al algoritmo sobre los que trabajar. El estado inicial aleatorio puede omitir algunas conexiones importantes para esa disposición concreta de sensores y centros de datos.


