\subsection{Experimento 6}

\paragraph{Enunciado}

Suponiendo que los centros de datos no sean costosos, podríamos estimar como
afecta el añadir más centros al coste de la red. Fijando el numero de sensores
en 100, realizad experimentos aumentando el
número de centros de datos de dos en dos hasta 10 y medid el coste de la 
red de conexión, el número de centros de datos usados y el coste temporal 
para hallar la solución. Usad el algoritmo de Hill Climbing y el
de Simulated Annealing.

\paragraph{Condiciones del experimento}

Para realizar este experimento necesitamos generar los datos correspondientes
a un total de 5 ejecuciones bajo diferentes criterios: 2 centros, 100 sensores;
4 centros, 100 sensores; 6 centros, 100 sensores; 8 centros, 100 sensores;
10 centros, 100 sensores;

Para garantizar que nuestra solucion es correcta, realizaremos 10 ejecuciones
por cada caso y recogeremos varios datos, sobre los cuales trataremos con
su mediana aritmetica. A parte, para obtener una
muestra representativa, generaremos las semillas para cada ejecución de manera
aleatoria.

Podemos resumir las características del experimento en la siguiente tabla:

\begin{tabular}{ | p{0.4\textwidth} | p{.6\textwidth} | }
  \hline
  Observación & Nuestra solución puede mejorar si aumentamos el numero de
  centros \\
  \hline
  Planteamiento & Escogemos diversos valores de numeros de centros en
  el estado, inicialmente 2 hasta llegar a 10\\
  \hline
  Hipótesis & La calidad de la solución mejora a mayor número de centros 
  tenemos (H0) o no\\
  \hline
  Método & 
    \begin{itemize}
      \item Escogemos semillas aleatoriamente para todas las repeticiones
      \item Ejecutamos 10 experimentos para cada configuracion de centros
        sensores que necesitamos
      \item Usamos el algoritmo de HillClimbing y el de Simmulated Annealing 
      \item Mediremos diferentes parámetros para realizar la comparación  
    \end{itemize}
    \\
  \hline
\end{tabular}

\paragraph{Resultados del experimento}

En las figuras \ref{fig:exp6-hill-cost}, \ref{fig:exp6-hill-time}, 
\ref{fig:exp6-simm-cost} y \ref{fig:exp6-simm-time} podemos observar
varios experimentos hechos con respecto al enunciado. Los resultados que
hablan de tiempo representan el cambio del tiempo de ejecución para
cada configuración de 2, 4, 6, 8, y 10 centros. Análogamente, los que
hablan de coste de conexión aluden al coste de la solución en esas
configuraciones.

\begin{figure}[h]
  \centering
  \includegraphics[scale=0.5]{exp6-hill-cost}
  \caption {Coste de la solución con el algoritmo Hill Climbing}
  \label{fig:exp6-hill-cost}
\end{figure}

\begin{figure}[h]
  \centering
  \includegraphics[scale=0.5]{exp6-hill-time}
  \caption {Tiempo de ejecución con el algoritmo Hill Climbing}
  \label{fig:exp6-hill-time}
\end{figure}

\begin{figure}[h]
  \centering
  \includegraphics[scale=0.5]{exp6-simm-cost}
  \caption {Coste de la solución con el algoritmo Simmulated Annealing}
  \label{fig:exp6-simm-cost}
\end{figure}

\begin{figure}[h]
  \centering
  \includegraphics[scale=0.5]{exp6-simm-time}
  \caption {Tiempo de ejecución con el algoritmo Simmulated Annealing}
  \label{fig:exp6-simm-time}
\end{figure}

\paragraph{Conclusiones}

El resultado del experimento se asemeja mucho a lo que nosotros nos esperabamos
que sucediese según la hipótesis nula H0. A excepción de Simmulated Annealing,
los resultados se vuelven más optimos a medida que aumentamos el numero de
centros de datos. Las soluciones aumentan de coste temporal a medida que
aumentamos el numero de centros, lo que también era de esperar.

