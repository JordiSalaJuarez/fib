\subsection{Implementación del estado}

El estado se ha implementado mediante un 
array de arrays de enteros \texttt{private int[][] connections} que indica
si un sensor está conectado a otro sensor o a un centro.

Las filas de esta estructura de datos indican, entre $[0, num\_sens]$\footnote{
$num\_sens$ indica el número de sensores del estado } el estado
de la conexion de los sensores. Las filas entre 
$(num\_sens, num\_sens + num\_centr]$\footnote{ $num\_centr$ indica el numero
de centros de datos del estado } indican el estado de conexion de los
centros.

Los sensores tan solo pueden estar conectados a otro elemento y pueden recibir
3 conexiones simultaneas, por lo que su estado de conexion es representado,
dentro de su fila, por un array \texttt{int[4]}, donde la posición 0 representa
a qué está conectado un sensor concreto, y las posiciones entre $[1, 3]$ 
representan qué otros sensores están conectados a ellos.

Los centros, por otro lado, ven su conexión representada por un array 
\texttt{int[25]}, donde cada posición entre $[0, 25]$ 
indica qué sensor está transmitiéndole datos.

\paragraph{}

Pensamos que esta es la mejor manera de implementar el problema,
ya que es bastante óptima en espacio y resulta muy eficiente a la
hora de operar con ella en opuesto a, por ejemplo, una 
implementación con clases definidas por nosotros.
