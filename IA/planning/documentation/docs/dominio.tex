\section{Descripción del dominio}

\subsection{Tipos}

Los tipos con los que contamos en el dominio son:

\begin{description}
  \item[\texttt{dia - object}:] Tipo dia, sus instancias representan 
    dias de la semana, por ejemplo \texttt{lunes}.
  \item[\texttt{plato - object}:] Tipo plato, sus instancias representan 
    platos de comida, por ejemplo \texttt{paella}.
  \item[\texttt{primero - plato}:] Tipo primero, hereda de plato. Sus 
    instancias representan platos que se consideran primeros, 
    por ejemplo \texttt{ensalada}.
  \item[\texttt{segundo - plato}:] Tipo segundo, hereda de plato. Sus 
    instancias representan platos que se consideran segundos, 
    por ejemplo \texttt{entrecot}.
  \item[\texttt{tipo\_plato - object}:] Tipo tipo de plato, sus instancias
    representan tipos de los que un plato puede ser,
    por ejemplo \texttt{pescado}.
\end{description}

\subsection{Predicados}

Los predicados con los que contamos en el dominio son:

\begin{description}
  \item[\texttt{(incompatible ?p - primero ?s - segundo)}:]

    Expresa que el primer plato \texttt{p} es incompatible con 
    el segundo \texttt{s}.
    Por ejemplo, \texttt{(incompatible paella lubina)} indica que
    \texttt{paella} y \texttt{lubina} no pueden ser servidas en el mismo dia
    \texttt{d} como primero y segundo respectivamente.

  \item[\texttt{(preparado ?p - plato)}:]

    Expresa que el plato \texttt{p} ya ha sido preparado en esta semana.
    Por ejemplo, \texttt{(preparado paella)} indica que
    \texttt{paella} ya ha sido preparado esta semana.

  \item[\texttt{(consecutivo ?d1 - dia  ?d2 - dia)}:]

    Expresa que el dia \texttt{d1} precede al dia \texttt{d2}.
    Por ejemplo, \texttt{(consecutivo lunes martes)} indica que
    \texttt{martes} es consecutivo a \texttt{lunes}.

  \item[\texttt{(ultimo\_dia ?d - dia)}:]

    Expresa que el dia \texttt{d} es el último dia de la semana.
    Por ejemplo, \texttt{(ultimo\_dia viernes)} indica que
    \texttt{viernes} es el último dia de la semana.

  \item[\texttt{(es\_de\_tipo ?p - plato ?t - tipo\_plato)}:]

    Expresa que 
    el plato \texttt{p} es de tipo \texttt{t}.
    Por ejemplo, \texttt{(es\_de\_tipo paella pescado)} indica que
    el plato \texttt{paella} es de tipo \texttt{pescado}.

  \item[\texttt{(tipo\_dia\_primero ?d - dia ?t - tipo\_plato)}:]

    Expresa que 
    en el dia \texttt{d} se ha servido un primer plato de tipo \texttt{t}.
    Por ejemplo, \texttt{(tipo\_dia\_primero jueves pescado)} indica que
    en el dia \texttt{jueves} se ha servido un primer plato
    de tipo \texttt{pescado}.

  \item[\texttt{(tipo\_dia\_segundo ?d - dia ?t - tipo\_plato)}:]

    Expresa que 
    en el dia \texttt{d} se ha servido un segundo plato de tipo \texttt{t}.
    Por ejemplo, \texttt{(tipo\_dia\_segundo martes sopa)} indica que
    en el dia \texttt{martes} se ha servido un segundo plato
    de tipo \texttt{sopa}.

  \item[\texttt{(servido\_primero ?d - dia)}:]

    Expresa que 
    se ha servido un primer plato el dia \texttt{d}.
    Por ejemplo, \texttt{(servido\_primero jueves)} indica que
    se ha servido un primer plato el dia \texttt{jueves}.

  \item[\texttt{(servido\_segundo ?d - dia)}:]

    Expresa que 
    se ha servido un segundo plato el dia \texttt{d}.
    Por ejemplo, \texttt{(servido\_segundo martes)} indica que
    se ha servido un segundo plato el dia \texttt{martes}.

  \item[\texttt{(menu\_primero ?d - dia ?p - primero)}:]

    Expresa que 
    se ha servido un primer plato \texttt{p} el dia \texttt{d}.
    Por ejemplo, \texttt{(menu\_primero jueves paella)} indica que
    se ha servido un primer plato \texttt{paella} el dia \texttt{jueves}.

  \item[\texttt{(menu\_segundo ?d - dia ?s - segundo)}:]

    Expresa que 
    se ha servido un segundo plato \texttt{p} el dia \texttt{d}.
    Por ejemplo, \texttt{(menu\_segundo martes fabada)} indica que
    se ha servido un segundo plato \texttt{fabada} el dia \texttt{martes}.

\end{description}

\subsection{Funciones}

\begin{description}
  \item[\texttt{(calorias ?p - plato)}:]
    Función que guarda el valor calórico del plato \texttt{p}.
  \item[\texttt{(precio\_total)}:]
    Función que guarda el precio total del menú semanal.
  \item[\texttt{(precio\_plato ?p - plato)}:]
    Función que guarda el precio de un plato \texttt{p}.
\end{description}

\subsection{Acciones}

\begin{description}

  \item[\texttt{servir\_ultimo\_dia\_primero}:]

    De manera resumida, esta accion
    sirve el primer plato del último dia de la semana.
    \\

    Esta accion se puede ejecutar
    bajo la condicion de que
    \begin{itemize}
      \item El primer plato del último dia de la semana no haya sido servido
    \end{itemize}
    provocando que
    \begin{itemize}
      \item El primer plato del último dia de la semana sea servido
      \item El precio total del menu semanal se vea incrementado por el precio 
        del plato a servir
    \end{itemize}

  \item[\texttt{servir\_ultimo\_dia\_segundo}:]

    De manera resumida, esta accion
    sirve el segundo plato del último dia de la semana.
    \\

    Esta accion se puede ejecutar
    bajo la condicion de que
    \begin{itemize}
      \item El segundo plato del último dia de la semana no haya sido servido
    \end{itemize}
    provocando que
    \begin{itemize}
      \item El segundo plato del último dia de la semana sea servido
      \item El precio total del menu semanal se vea incrementado por el precio 
        del plato a servir
    \end{itemize}

  \item[\texttt{servir\_primero\_solo}:]

    De manera resumida, esta accion
    sirve el primer plato de un dia de la semana cuando su segundo
    ya ha sido servido.
    \\
    
    Esta accion se puede ejecutar
    bajo la condicion de que
    \begin{itemize}
      \item Ni el primer ni el segundo plato del dia a servir hayan sido
        servidos
      \item El plato a servir no haya sido servido en la semana actual
      \item El primer plato servido el dia consecutivo al dia a servir sea de
        un tipo distinto del del plato a servir
    \end{itemize}
    provocando que
    \begin{itemize}
      \item El primer plato del dia a servir sea servido
      \item El precio total del menu semanal se vea incrementado por el precio 
        del plato a servir
    \end{itemize}

  \item[\texttt{servir\_primero}:]

    De manera resumida, esta accion
    sirve el primer plato de un dia de la semana cuando su segundo
    aún no ha sido servido.
    \\

    Esta accion se puede ejecutar
    bajo las condiciones de que:
    \begin{itemize}
      \item El primer plato del dia a servir no haya sido servido
      \item El segundo plato del dia a servir haya sido servido
      \item El plato a servir no haya sido servido en la semana actual
      \item El primer plato servido el dia consecutivo al dia a servir sea de
        un tipo distinto del del plato a servir
      \item El plato a servir y el segundo plato del dia a servir no sean
        incompatibles
      \item La suma de las calorias del plato a servir y del segundo plato
        del dia a servir no debe ser menor a 1000 ni mayor a 1500
    \end{itemize}
    provocando que:
    \begin{itemize}
      \item El primer plato del dia a servir sea servido
      \item El precio total del menu semanal se vea incrementado por el precio 
        del plato a servir
    \end{itemize}

  \item[\texttt{servir\_segundo}:]

    De manera resumida, esta accion
    sirve el segundo plato de un dia de la semana cuando su primero
    aún no ha sido servido.
    \\

    Esta accion se puede ejecutar
    bajo las condiciones de que:
    \begin{itemize}
      \item El segundo plato del dia a servir no haya sido servido
      \item El primer plato del dia a servir haya sido servido
      \item El plato a servir no haya sido servido en la semana actual
      \item El segundo plato servido el dia consecutivo al dia a servir sea de
        un tipo distinto del del plato a servir
      \item El plato a servir y el primer plato del dia a servir no sean
        incompatibles
      \item La suma de las calorias del plato a servir y del primer plato
        del dia a servir no debe ser menor a 1000 ni mayor a 1500
    \end{itemize}
    provocando que:
    \begin{itemize}
      \item El segundo plato del dia a servir sea servido
      \item El precio total del menu semanal se vea incrementado por el precio 
        del plato a servir
    \end{itemize}

\end{description}
